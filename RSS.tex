\documentclass{article}
\usepackage{graphicx}
\usepackage[margin=1in]{geometry}
\usepackage{amsmath}
\usepackage{amsfonts}
\usepackage{amsthm}
\newtheorem{theorem}{Theorem}
\usepackage{hyperref}
\title{Discrete Random Subset Sum Problem}
\author{Aakash Kumar}
\date{June 2025}

\begin{document}

\maketitle

\section{The Random Subset Sum Problem}
Consider the set of random variables $X_1,X_2,\dots,X_n$ sampled uniformly from $\{-M,-M+1,\dots,M-1,M\}$ for some positive integer $M$ and target $z\in\{-M,-M+1,\dots,M-1,M\}$. Let $e\in\mathbb{N}\cup\{0\}$. The Random Subset Sum Problem (RSSP) is the problem of finding a subset $S\subseteq [n]$ such that
\begin{equation*}
    \big|\sum_{i\in S} X_i - z\big|\leq e.
\end{equation*}
Let $f_t:\mathbb{Z}\to\{0,1\}$ be the indicator for the event $z$ is $e$ approximated at time $t\in\mathbb{N}$, i.e.,
\begin{equation*}
    f_t(z) = \mathbb{I}_{\exists S\subseteq [t]\;:\;|\sum_{i\in S}X_i-z|\leq e}.
\end{equation*}
It is clear that $f_t$ follows the following recurrence relation
\begin{equation*}
    f_{t+1}(z)=f_{t}(z)+(1-f_t(z))f_t(z-X_{t+1}).
\end{equation*}
Define $v_t$ as
\begin{equation*}
    v_t=\sum_{i=-M}^M f_t(i).
\end{equation*}
\begin{theorem}
\label{thm:EVt}
For all $0\leq t \leq n$, it holds that
\begin{equation*}
    \mathbb{E}(v_{t+1}|X_1,X_2,\dots,X_t)\geq v_t+\frac{v_t}{4}\left(1-\frac{v_t}{2M}\right).
\end{equation*}
\end{theorem}
\begin{proof}
Consider
\begin{align*}
    \mathbb{E}(v_{t+1}|X_1,X_2,\dots,X_t) &= \mathbb{E}\left(\sum_{i=-M}^M f_{t+1}(i)|X_1,X_2,\dots,X_t\right)\\
    &= \mathbb{E}\left(\sum_{i=-M}^M (f_{t}(i)+(1-f_t(i))f_t(i-X_{t+1}))|X_1,X_2,\dots,X_t\right)\\
    &= \frac{1}{2M}\sum_{j=-M}^M \sum_{i=-M}^M f_t(i)+(1-f_t(i))f_t(i-j)\\
    &= \sum_{i=-M}^M f_t(i)+\frac{1}{2M}\sum_{j=-M}^M \sum_{i=-M}^M (1-f_t(i))f_t(i-j)\\
    &= v_t+\frac{1}{2M} \sum_{i=-M}^M \left((1-f_t(i))\sum_{j=-M}^M f_t(i-j)\right)\\
    &= v_t+\frac{1}{2M} \sum_{i=-M}^M (1-f_t(i))\sum_{k=i-M}^{i+M}f_t(k)
    \end{align*}
where $k=i-j$. Hence we have
    \begin{align*}    
    \mathbb{E}(v_{t+1}|X_1,X_2,\dots,X_t) \geq v_t + \frac{1}{2M}\sum_{i=-\frac{M}{2}+\mu}^{i=\frac{M}{2}+\mu}\left((1-f_t(i))\sum_{k=-\frac{M}{2}+\mu}^{k=-\frac{M}{2}+\mu}f_t(k)\right)
\end{align*}
for some $\mu\in \{-M/2,\dots,M/2\}$. Now $\exists\; \mu^*\in \{-M/2,\dots,M/2\}$ such that
\begin{equation*}
    \sum_{k=-\frac{M}{2}+\mu}^{k=-\frac{M}{2}+\mu}f_t(k)\geq \frac{v_t}{2}.
\end{equation*}
Choose $\mu=\mu^*$. Hence we get
\begin{align*}
    \mathbb{E}(v_{t+1}|X_1,X_2,\dots,X_t) &\geq v_t + \frac{1}{2M}\frac{v_t}{2}\sum_{i=-\frac{M}{2}+\mu}^{i=\frac{M}{2}+\mu}\left(1-f_t(i)\right)\\
    &\geq v_t+\frac{v_t}{4}\left(1-\frac{v_t}{2M}\right).
\end{align*}
\end{proof}
\noindent The main results depend on analysis how $v_t$ grows with $t$. It turns out that $v_t$ grows quickly to $M$ and then slowly rises to $2M$. Define $\tau_1$ as
\begin{equation*}
    \tau_1 = \min \{t\geq 0: v_t>M\}.
\end{equation*}
\begin{theorem}
    \label{thm:pvt1}
    Given $\beta \in (0,1/8)$, let $p_\beta = 1-\frac{7}{8(1-\beta)}$. For all integers $0\leq t<\tau_1$ it holds that
    \begin{equation*}
        \mathbb{P}(v_{t+1}\geq v_t(1+\beta)\;|\;X_1,\dots,X_t,t\leq \tau_1)\geq p_\beta.
    \end{equation*}
\end{theorem}
\begin{proof}
    Define the process
    \begin{equation*}
            \tilde{v}=\sum_{i=-M}^M(f_t(i)+(1-f_t(i))f_t(i-X_{t+1})\mathbb{I}_{\{M,\dots,M\}})
    \end{equation*}
    Clearly $\tilde{v}\leq v_{t+1}$ and $\tilde{v}\leq 2v_t$. The bound in Theorem \ref{thm:EVt} also holds for $\tilde{v}$. Hence we have
    \begin{equation*}
        \mathbb{P}(v_{t+1}\geq v_t(1+\beta)\;|\;X_1,\dots,X_t,t\leq \tau_1)\geq \mathbb{P}(\tilde{v}\geq v_t(1+\beta)\;|\;X_1,\dots,X_t,t\leq \tau_1).
    \end{equation*}
    Now recall the reverse Markov's inequality, for any random variable $X$ such that $0\leq X\leq b$ and $0\leq a\leq\mathbb{E}[X]$, we have
    \begin{equation*}
        \mathbb{P}(X\geq a)\geq \frac{\mathbb{E}[X]-a}{b-a}.
    \end{equation*}
    Using reverse Markov's inequality, we have
    \begin{align*}
        \mathbb{P}(\tilde{v}\geq v_t(1+\beta)\;|\;X_1,\dots,X_t,t\leq \tau_1) &\geq \frac{\mathbb{E}[v_t|X_1,\dots,X_t,t<\tau_1]-v_t(1-\beta)}{2v_t-v_t(1-\beta)}\\
        & \geq \frac{\frac{9}{8}v_t-v_t(1+\beta)}{2v_t-v_t(1+\beta)}\\
        &=\frac{\frac{9}{8}-(1+\beta)}{1-\beta}\\
        &=1-\frac{7}{8(1-\beta)}.
    \end{align*}
\end{proof}
\end{document}
